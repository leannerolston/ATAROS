\section{Introduction}

Goals:  To find non non-prosodic markers of stance (lexical, ``stylistic'') 

- Does LSM vary with stance strength 

Can stance be predictable from function word use?  
Does this perform better than unigrams?  
Can either generalize to a different corpus?  



Are there specific stylistic choices made in expressing stance.  If it can be boiled down to specific stylistic words, or classes of stylistic words, it rather than specific stancy words, it might be more readily applied to a multilingual context.

\subsection{What is Stance}

Stance is an important part of human communication.  From the most basic small talk, to expressing basic needs and wants, we are expressing stance.   Stance is formally defined as the ``expression of ones subjective attitude'' toward a target \citep{haddington2004stance}, be it an entity or a proposition.  Subjective attitude includes, among other things, personal feelings, attitude, opinion, evaluation, and belief.  It refers to a speaker or writer's private or internal state that is ``not open to objective observation or verification'' \citep{wiebe2006word, wiebe2004learning}.  Subjectivity is contrasted with objectivity, which is the presentation of information as factual, which is not to say that subjectivity is inherently not factual, nor that objective statements are inherently true.  

Stance can be expressed in various ways, from word choice, grammatical structures, and paralinguistically in spoken interactions, through devices such as pitch, loudness, blah blah.. 

Broadly speaking, stance can be broken into three semantic categories: epistemic, attitudinal, and style of speaking \citep{biber1999grammar}.  Epistemic stance refers to the status of the information being given, such as the certainty or doubt (probably, definitely), precision or limitation (exactly, roughly), or even the source of the information (according to, it is common knowledge that..).  Attitudinal or evaluative stance, on the other hand, reveals the speaker's personal feelings.  A third classification of stance, style of speaking stance, refers to the way the information is being presented (honestly, I swear).  

All of these forms of stance can be expressed directly or indirectly.  The acoustic signal can carry a negative meaning in otherwise lexically positive utterances.  \cite{freeman2015prosody}  found 61 instances of the token ``yeah'' serving a negative function in the ATAROS corpus, and annotators with access to the recordings ''annotate better' than those who do not (CITE).  Sarcasm is another means to alter the lexical meaning of an expression.  Additionally, there are instances where one does not want to openly share their stance. It is instances such as these that I hope uncover through this research.  

There has been a lot of work in what you reveal through your use of function words.  Specific patterns of function word use have been linked to the psychological states of depression \citep{chung2007psychological}, those relating to deception (cite), the likelihood of carrying out a threat \citep{gales2017threatening}... 
%The detection of stance and other private states falls under the umbrella of Subjectivity Analysis.  The use of the term \emph{stance detection} in this paper, is not to be confused with the field referred to as \emph{Stance Detection} which usually refers to the detection of stance \emph{polarity}.  
%
%Stance can be expressed using single words, multi-word expressions, syntactic and morphological devices such as changes in aspect \cite{wiebe2004learning}.  Subjectivity is a associated with word senses, if not the words themselves, since many subjective words and phrases have an alternative literal or objective meaning \cite{wiebe2006word}. In spoken contexts, in addition to the words chosen, stance is expressed through the speech signal, even to the point where the acoustic signal contradicts and overrides the word-sense subjectivity \cite{freeman2015phonetics, freeman2015prosody}.  Stance-expressing statements were found to be be hyperarticulated relative to statements that do not express stance \cite{freeman2014hyperarticulation}, with  ... 
%
%
%\begin{table}[h]
%\centering
%\begin{tabular}{l | l} 
%	\textbf{Subjective} & \textbf{Objective} \\ 
%	\hline 
%	subjective statement & objective statement \\ 	
%\end{tabular}
%\caption{Grammatical and Lexical Structures in their Subjective and Objective Use} 
%\label{tab:subjective_vs_objective_examples} 
%\end{table}
 
%Stance taking == dialogical and intersubjective activity where subjects rely on multiple linguistic resources and interactional practices 
%- aka modality, evaluation, subjectivity, epistemicity, footing, alignment, assesment, agreement to refer to the speaker's attitude, displays of emotion and desires, expressions of beliefs and certainty toward given issues, people, and the speaker's co-participants. 
%for single speaker-speaker contributions to stance the starting point is usually a linguistic form (evaluative lexical item of stance-frame) 

DUBOIS Stance triangle.

Dialogic 

Biber 

People can better transcribe stance when the audio is available. 


\subsection{Previous Work}

The majority of previous work on Subjectivity Analysis relies on the collection of a lexicon 

MPQA

Biber 

Opinion Mining 

Wiebe learing from text
 
 Necessary pre-req to Stance detection 